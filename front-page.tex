\normalsize
\begin{tikzpicture}
\newcount\mycount

% More ideas
% moon phases
% Percent of year over
% week number (and total)
% seasons
% Julian day

\pgfcalendardatetojulian{1991-07-01}{\mycount} %Days old
\edef\birthday{\the\mycount}
\pgfcalendardatetojulian{\weekstart+-1}{\mycount}
\advance\mycount by -\birthday
\edef\daysold{\the\mycount}

\pgfcalendarifdate{\weekstart}{at most=\weekyear-07-01}
{\pgfcalendardatetojulian{\weekyear-07-01}{\mycount}}
{\mycount=\weekyear
  \advance\mycount by 1
  \pgfcalendardatetojulian{\mycount-07-01}{\mycount}}
\edef\nextbirthday{\the\mycount}
\pgfcalendardatetojulian{\weekstart+-1}{\mycount}
\advance\mycount by -\nextbirthday
\multiply\mycount by -1
\edef\daysuntilbday{\the\mycount} %Days until next bday

\draw (0,0) node [anchor=north west, xscale=1]{
M\=oller-Mara \scriptsize{\daysold/\daysuntilbday}
};
\draw (0,-0.6cm) node [anchor=north west, text width=\width, xscale=0.75, yscale=1, transform shape]{
\normalsize{\texttt{DIE}}
} ++(0, -0.3cm) node [anchor=north west, text width=\width, text centered, xscale=0.75, yscale=1, transform shape]{
\Huge{\textbf{\texttt{ZEITPLANUNGMASCHINE}}}\\
\scriptsize{\textit{SCHEDULING ENGINE}}
};


\newcount\daysinmonth
\newcount\daysinyear
\pgfcalendardatetojulian{\weekyear-\weekmonth-last+1}{\daysinmonth}
\pgfcalendardatetojulian{\weekyear-\weekmonth-01}{\mycount}
\advance \daysinmonth by -\mycount
\pgfcalendardatetojulian{\weekyear-12-last+1}{\daysinyear}
\pgfcalendardatetojulian{\weekyear-01-01}{\mycount}
\edef\startyear{\the\mycount}
\advance \daysinyear by -\mycount


\newdimen\rotdiff
\newcount\total
\newcount\temp
\let\pgfcsh=\pgfcalendarshorthand

\tikzstyle{filling}=[fill=blue!20]
\begin{scope}[xshift=92pt, yshift=-150pt]
  \begin{scope}
        \calendar (large_cal) at (0,0) [ dates=\weekyear-01-01 to \weekyear-12-last,
        day code={
          \draw (\the\rotdiff:-0.25cm) node (circ-\pgfcsh{y}{0}-\pgfcsh{m}{0}-\pgfcsh{d}{0}) [filling, circle, minimum size=2pt] {};
          \ifdate{day of month = 1}
          {\node[name=\pgfcalendarsuggestedname,every day, anchor=center]{\scriptsize{\texttt{\pgfcalendarshorthand{m}{.}}}};}{}}]
    if (at most=\weekstart-1) {
      \tikzstyle{filling}=[fill=red]
    }
    if (at least=\weekstart+7) {
      \tikzstyle{filling}=[fill=gray!80]
    }
    if (between=\weekstart and \weekstart+6) {
      \tikzstyle{filling}=[fill=blue]
    } 
    if (day of month = 1) {
      \tikzstyle{filling}=[fill=black]
    }
    if (all) {
      \total=\pgfcalendarendjulian
      \advance\total by -\pgfcalendarbeginjulian
      \advance\total by 1
      \temp = \pgfcalendarcurrentjulian
      \advance\temp by -\pgfcalendarbeginjulian      
      \pgfmathsetlength\rotdiff{ \temp * -1 * (360 / \total) + 90}
      \pgftransformshift{\pgfpointpolar{\rotdiff}{2.2cm}} 
    };
    % \node [above of=large_cal-\weekyear-01-01] {\weekyear};
  \end{scope}
  \begin{scope}
    \calendar at (0,0) [ dates=\weekyear-\weekmonth-01 to \weekyear-\weekmonth-last, day
    code={
      \node[name=\pgfcalendarsuggestedname,every
      day, anchor=center]{\scriptsize{\texttt{\tikzdaytext}}};
    }] 
    if (at most=\startdate-1) [red!50!black]
    if (at least=\startdate+1) [gray!80]
    if (Sunday) [black]
    if (between=\startdate and \startdate) [blue] %How do I say ``equal'' again??
    if (day of month=1) { 
      \draw (0,0) node (month) [anchor=north,blue]
      {\large{\texttt{\pgfcalendarshorthand{m}{t}}}};
      \draw node [above of=month,node distance=.5cm, anchor=north, black] {\large{\texttt{\pgfcalendarshorthand{y}{0}}}}; }
    if (between=\startdate and \startdate) {
      \draw node [below of=month,node distance=.5cm, anchor=south, black] {\large{\texttt{\pgfcalendarshorthand{d}{0}}}};
    }
    if (all) {
      \pgfmathsetlength\rotdiff{ (-\the\numexpr \pgfcalendarcurrentday \relax + 1 )* 360 / %Remove leading zeros
        \daysinmonth + 90}
      \pgftransformshift{\pgfpointpolar{\rotdiff}{1.5cm}} };
  \end{scope}
  \begin{scope}
    \calendar at (0,0) [ dates=\startdate to \startdate+6, day
    code={
        \pgfcalendardatetojulian{\startdate}{\mycount}
        \temp = \pgfcalendarcurrentjulian
        \advance\temp by -\mycount
        \node [anchor=center, black] at (\the\rotdiff:-0.25cm) {\scriptsize{\texttt{\shortweekday{\the\temp}}}};
    }] 
    if (equals=\weekyear-\weekmonth-last+1) {
        \draw (0,0) node (tomonth) [below of=month, node distance=.25cm] {\small{\texttt{to}}};
        \node (0,0) (nextmonth) [below of=tomonth, node distance=.25cm] {\small{\texttt{\pgfcalendarshorthand{m}{.}}}};
    }
    if (equals=\weekyear-12-last+1) {
        \node [below of=nextmonth, node distance=.25cm] (0,0) {\small{\texttt{\pgfcalendarshorthand{y}{0}}}};
    }
    if (all) {
      \pgfmathsetlength\rotdiff{ (-\the\numexpr \pgfcalendarcurrentday \relax + 1 )* 360 /
        \daysinmonth + 90}
      \pgftransformshift{\pgfpointpolar{\rotdiff}{1.5cm}} };
  \end{scope}

  \newcommand{\blahdude}[2]
  {
    \pgfcalendardatetojulian{#1}{\temp}
    \advance\temp by -\startyear
    \edef\firstday{\the\temp}
    \pgfcalendardatetojulian{#2}{\temp}
    \advance\temp by -\startyear
    \pgfmathsetlength\rotdiff{ abs(\temp + \firstday) / 2 * -1 * (360 / \daysinyear) + 90}
  }
  \newcommand{\findpercent}[2]
  {
    \pgfcalendardatetojulian{#1}{\temp}
    \advance\temp by -\startyear
    \edef\firstday{\the\temp}
    \pgfcalendardatetojulian{#2}{\temp}
    \advance\temp by -\startyear
    \pgfmathparse{ (\temp - \firstday) * (100 / \daysinyear)}
  }
  \newcommand{\finddiff}[2]
  {
    \pgfcalendardatetojulian{#1}{\temp}
    \advance\temp by -\startyear
    \edef\firstday{\the\temp}
    \pgfcalendardatetojulian{#2}{\temp}
    \advance\temp by -\startyear
    \advance\temp by -\firstday
  }
  \blahdude{\weekyear-01-01}{\weekstart+0}
  \edef\pasttmiddle{\the\rotdiff}
  % \draw (0,0) -- (\pasttmiddle:2.5cm);
  \blahdude{\weekstart}{\weekstart+6}
  \edef\presentmiddle{\the\rotdiff}
  % \draw (0,0) -- (\presentmiddle:2.5cm);
  \blahdude{\weekstart+6}{\weekyear-12-31}
  \edef\futuremiddle{\the\rotdiff}
  % \draw [thick, green] (0,0) -- (\futuremiddle:2.5cm);

  % \draw [ultra thick, orange] (-100pt, 100pt) -- (-100pt, -100pt) coordinate (linea);
  % \fill [green] (\the\rotdiff:2.5cm -| linea) circle (2pt);

  \newdimen\outerring
  \outerring=2.8cm
  \newdimen\innerring
  \innerring=2.5cm
  \coordinate (futurestat) at (-65pt, -100pt);
  \coordinate (paststat) at (65pt, -100pt);
  \draw [-o, thick] (futurestat) -- ++(-20pt, 10pt) |- (\futuremiddle:\outerring) -- (\futuremiddle:\innerring);
  \draw [-o, thick] (paststat) -- ++(20pt, 10pt) |- (\pasttmiddle:\outerring) -- (\pasttmiddle:\innerring);

  \tikzstyle{stat info}=[text width=40pt, text centered, anchor=north]
  \findpercent{\weekyear-01-01}{\weekstart+0}
  \edef\pastpercent{\pgfmathresult}
  \finddiff{\weekyear-01-01}{\weekstart+0}
  \draw (paststat) ++ (0, -2pt) node (pasttwo) [stat info] {\texttt{\scriptsize{\pastpercent\%}}};
  \draw (pasttwo.south) ++ (0, -1pt) node [stat info] {\texttt{\scriptsize{\the\temp}}};

  \findpercent{\weekstart+6}{\weekyear-12-31}
  \edef\futurepercent{\pgfmathresult}
  \finddiff{\weekstart+6}{\weekyear-12-31}
  \draw (futurestat) ++ (0, -2pt) node (futuretwo) [stat info] {\texttt{\scriptsize{\futurepercent\%}}};
  \draw (futuretwo.south) ++ (0, -1pt) node [stat info] {\texttt{\scriptsize{\the\temp}}};

  \draw (0, -100pt) node (presenttwo) [stat info] {\texttt{\scriptsize{\weekjulianday}}};
  \draw (presenttwo.south) ++ (0, -1pt) node [stat info] {\texttt{\scriptsize{W\weeknum}}};

\end{scope}
\newcommand{\daysuntil}[1]
{
\pgfcalendardatetojulian{\weekstart+-1}{\mycount}
\edef\temp{\the\mycount}
\pgfcalendardatetojulian{#1}{\mycount}
\advance\mycount by -\temp
}

\end{tikzpicture}

%%% Local Variables: 
%%% mode: latex
%%% TeX-master: "zeitplanungmaschine.tex"
%%% End: 
