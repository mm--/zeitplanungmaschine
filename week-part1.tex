\scriptsize
    \begin{tikzpicture}
      \def\Tiny{\fontsize{8pt}{8pt}\selectfont}
      \pgfmathsetlength\width{\width - 2pt}
      \tikzstyle{eventstyle}=[color=black, text width=\width]

      \newcount\eventcount
      \newcount\temp
      \renewcommand{\blah}[1]
      {
        \edef\argh{event\pgfcalendarcurrentday-\the\eventcount}
        \advance\eventcount by 1
        \ifthenelse{\equal{\the\eventcount}{1}}
        {  \draw (event\pgfcalendarcurrentday.south west) ++ (0, -4pt) node (event\pgfcalendarcurrentday-\the\eventcount) [eventstyle] {#1};}
        {  \draw (\argh.south) ++(0, -2pt) node (event\pgfcalendarcurrentday-\the\eventcount) [anchor=north, eventstyle] {#1};}%
      }

      \renewcommand{\reminder}[8]
      {
        \pgfmathsetcount\temp{mod(\pgfcalendarcurrentweekday+1,7)}
        \ifdate{equals={#1}-{#2}-{#3}}
        {
          \blah{#5 \Tiny{#6}}
        }{}
      }

      \setcounter{weekypos}{0}

      \pgfcalendar{cal}{\weekstart}{\weekstart+2}
                  {%
                    \leavevmode%
                    \let\%=\pgfcalendarshorthand
                    \addtocounter{weekypos}{-3}
                    \ordinaldate{\pgfcalendarcurrentjulian}{\pgfcalendarcurrentyear}
                    \draw (0, \value{weekypos}) node (event\pgfcalendarcurrentday) [daylabel, text width=\width] {\%wt --- \%m. \%d0 (\firstord/\secondord)};
                    \eventcount=0
                    % \reminder{2011}{01}{17}{MLK}{Martin Luther King Jr. Day}{}{}{}
\reminder{2011}{02}{14}{Valentines}{Valentines}{}{}{}
\reminder{2011}{10}{31}{Halloween}{Halloween}{}{}{}
\reminder{2011}{12}{25}{Christmas}{Christmas}{}{}{}
\reminder{2011}{05}{30}{Memorial Day}{Memorial Day}{}{}{}
\reminder{2011}{03}{31}{Ch�vez}{C�sar Ch�vez Day}{}{}{}
\reminder{2011}{07}{04}{}{Independence Day}{}{}{}
\reminder{2011}{09}{05}{Labor Day}{Labor Day}{}{}{}
\reminder{2011}{10}{10}{}{Columbus Day}{}{}{}
\reminder{2011}{11}{11}{Veterans}{Veterans Day}{}{}{}
\reminder{2011}{11}{24}{Thanksgiving}{Thanksgiving Day}{}{}{}
\reminder{2011}{02}{02}{}{Groundhog Day}{}{}{}
\reminder{2011}{02}{03}{}{Chinese New Year}{}{}{}
\reminder{2011}{03}{17}{}{Saint Patrick's Day}{}{}{}
\reminder{2011}{05}{08}{Mother's Day}{Mother's Day}{}{}{}
\reminder{2011}{06}{19}{Father's Day}{Father's Day}{}{}{}

%%% Local Variables: 
%%% mode: latex
%%% TeX-master: "zeitplanungmaschine.tex"
%%% End: 

% Birthdays are entered in here
\birthday{01}{18}{Steve McFakeName}{}
\birthday{07}{27}{John Nobody}{1950}

%%% Local Variables: 
%%% mode: latex
%%% TeX-master: "zeitplanungmaschine.tex"
%%% End: 


\reminder{2013}{07}{28}{Event 1}{Event 1}{}{}{}{}
\reminder{2013}{07}{20}{Event 2}{Event 2}{}{}{}{}
\reminder{2013}{07}{19}{Event 3}{Event 3 is going to be very exciting}{}{13:30}{}{}

\input{testdates.tex}

\lesserreminder{2013}{07}{18}{Stuff}{Blah!}{}{}{}

%%% Local Variables: 
%%% mode: latex
%%% TeX-master: "zeitplanungmaschine.tex"
%%% End: 

                  }

    \end{tikzpicture}

%%% Local Variables: 
%%% mode: latex
%%% TeX-master: "zeitplanungmaschine.tex"
%%% End: 
