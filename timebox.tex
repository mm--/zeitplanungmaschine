\scriptsize
\begin{tikzpicture}
  \def\Tiny{\fontsize{3pt}{3pt}\selectfont}
  \newdimen\newx
  \newdimen\newy
  \pgfmathsetlength\newx{\width / 8}
  \pgfmathsetlength\newy{(\height - 2) / 25}
  \renewcommand{\dispevent}[9]
  {
    \ifthenelse{\equal{#7}{}}{}{
      \newdimen\timespandimen
      \parseTime{#7}
      \edef\timeone{\pgfmathresult}
      \ifthenelse{\equal{#8}{}}{\parseTime{#7}\pgfmathparse{\pgfmathresult+1}}{\parseTime{#8}}
      \edef\timetwo{\pgfmathresult}
      \pgfmathsetlength\timespandimen{(\timetwo - \timeone) * \newy}
      \begin{scope}
        \clip (-1, -\timeone) rectangle ++(\value{loopindex}+3,-\timespandimen);
        \draw (\value{loopindex}, -\timeone) rectangle ++(1,-\timespandimen+0.1pt);
        \draw (\value{loopindex}, -\timeone) node [draw=none, minimum height=\timespandimen, fill=yellow, fill opacity=0.4, text
        opacity=1] {#5};
      \end{scope}
    }
  }

    % \ifthenelse{\equal{#7}{}}{}
    % {\draw (\pgfcalendarcurrentweekday, \parsetime{#7}) node [minimum height=1*\newy, fill=red!80] {};}

  \begin{scope}[x=\newx, y=\newy]
    \tikzstyle{thefill}=[]
    \tikzstyle{every node}=[text width=\newx, text centered, minimum height=\newy, thefill]
    % \pgfcalendar{cal}{\theyear-\themonth-\theday-1}{\theyear-\themonth-\theday+5}{
    %   \addtocounter{loopindex}{1}
    %   \foreach \time in {-1, 0,..., 23}
    %   {
    %     \ifodd \loopindex \def\filltime{white}\else\def\filltime{green}\fi
    %     \ifodd \time \edef\filltime{\filltime}\else\def\filltime{blue}\fi
    %     \tikzstyle{thefill}=[fill=\filltime!10]
    %     \draw (\loopindex, -\time) node {\ifthenelse{\time < 0} {\shortweekday{\day}}
    %       {\ifthenelse{\day < 0}{\time:00}{}}};
    %   }
    %   \ifthenelse{\loopindex < 0}{}{\draw [white, ultra thick] (\day, 1) -- (\day, -24);}
    % }
    \pgfcalendar{cal}{\theyear-\themonth-\theday}{\theyear-\themonth-\theday}
    {
      \pgfcalendarcurrentweekday

      \foreach \day in {-1, 0,..., 6}
      {
        \foreach \time in {-1, 0,..., 23}
        {
          \ifodd \day \def\filltime{white}\else\def\filltime{green}\fi
          \ifodd \time \edef\filltime{\filltime}\else\def\filltime{blue}\fi
          \tikzstyle{thefill}=[fill=\filltime!10]
          \pgfmathparse{int(mod(\day+\pgfcalendarcurrentweekday,7))} % Week begins on Sunday
          \let\&=\pgfmathresult
          \draw (\day, -\time) node {\ifthenelse{\time < 0} {\ifthenelse{\day < 0}{}{\shortweekday{\&}}}
            {\ifthenelse{\day < 0}{\time:00}{}}};
        }
        \ifthenelse{\day < 0}{}{\draw [white, ultra thick] (\day, 1) -- (\day, -24);}
      }
    }
    \tikzstyle{thefill}=[draw=black]
    \Tiny
    % \foreach \x in {0, 2, 4}
    % {\draw (\x, -13) node [minimum height=1*\newy, fill=black!20] {Stat 134};
    %   \draw (\x, -14) node [minimum height=1*\newy, fill=red!20] {Ling 100};}
    % \foreach \x in {0, 2} 
    % {\draw (\x, -16) node [minimum height=1.5*\newy, fill=purple!20] {CS 162};}

    % \foreach \x in {1, 3}
    % {
    %   \draw (\x, -12.5) node [minimum height=1.5*\newy, fill=green!20] {CS 188};
    %   \draw (\x, -14) node [minimum height=1.5*\newy, fill=blue!20] {Psych 128};
    % }

    % \draw (1, -9) node [minimum height=1*\newy, fill=red!20] {Ling Dis};
    % \draw (1, -10) node [minimum height=1*\newy, fill=purple!20] {CS162 Dis};
    % \draw (2, -15) node [minimum height=1*\newy, fill=green!20] {CS188 Dis};


    % \foreach \x in {1, 3}{
    % \draw (\x, -11) node [minimum height=1*\newy, fill=gray!20] {117};
    % \draw (\x, -8) node [minimum height=1.5*\newy, fill=green!20] {110};
    % \draw (\x, -9.5) node [minimum height=1.5*\newy, fill=blue!20] {MCB};
    % \draw (\x, -14) node [minimum height=1.5*\newy, fill=orange!20] {HofI};}

    % \draw (0, -13) node [minimum height=1*\newy, fill=blue!20] {MCB Dis};
    % \draw (3, -13) node [minimum height=1*\newy, fill=gray!20] {117 Dis};
    % \draw (2, -8) node [minimum height=1*\newy, fill=green!20] {110 Dis};
    
    \setcounter{loopindex}{-1}
    \pgfcalendar{cal}{\theyear-\themonth-\theday}{\theyear-\themonth-\theday+6}
    {
      \pgfcalendarcurrentweekday
      \addtocounter{loopindex}{1}
      \disptoday
    }

  \end{scope}
\end{tikzpicture}

%%% Local Variables: 
%%% mode: latex
%%% TeX-master: "zeitplanungmaschine.tex"
%%% End: 
