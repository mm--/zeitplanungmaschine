\newcommand{\reminder}[8]{}     %dummy command, gets renewed 
\newcommand{\blah}[1]{}       %more dummies

\newcommand{\dummybirthday}[4]{}

\newcommand{\shortweekday}[1]%
{%
\ifthenelse{\equal{#1}{0}}{M}{}%
\ifthenelse{\equal{#1}{1}}{Tu}{}%
\ifthenelse{\equal{#1}{2}}{W}{}%
\ifthenelse{\equal{#1}{3}}{Th}{}%
\ifthenelse{\equal{#1}{4}}{F}{}%
\ifthenelse{\equal{#1}{5}}{Sa}{}%
\ifthenelse{\equal{#1}{6}}{Su}{}%
}

% Key/Value pairs for events
\gdef\setkey#1{\expandafter\gdef\csname myarray(#1)\endcsname} 
\gdef\dispkeyval#1{\csname myarray(#1)\endcsname}

\let\ea=\expandafter            %Don't overwrite this
\newcommand{\appendtokey}[2]
{
  % I swear if this ever breaks I'll have no idea how to fix it
  % Also this is ugly as hell, but I can't find a better way to do it
  \ea\ea\ea\def\ea\ea\ea\nameofarray\ea\ea\ea{\dispkeyval{#1}}
  \ea\ea\ea\def\ea\ea\ea\valueofarray\ea\ea\ea{\nameofarray}
  \ea\def\ea\newvalueofarray\ea{\valueofarray #2}
  \ea\ea\ea\gdef\ea\nameofarray\ea{\newvalueofarray}
}

\newcommand{\addtokey}[2]
{
  \ifcsname myarray(#1)\endcsname
  \appendtokey{#1}{#2}
  \else
  \setkey{#1}{#2}
  \fi
}


\newcommand{\formattime}[2]
{\texttt{\ifthenelse{\equal{#1}{}}{}{[#1\ifthenelse{\equal{#2}{}}{}{-#2}]}}}

\def\timetounits(#1:#2){%
  \def\a{#1}
  \def\b{#2}
  % a is \a\\
  % b is \b\\
  \pgfmathparse{\a + (\b/60)}
}
\newcommand{\parseTime}[1]
{\expandafter\timetounits(#1)}
\def\parsetime{\parseTime}

\newcommand{\dispevent}[8]{}
\def\displesserevent{\dispevent}
\def\dispweekreminder{\dispevent}

\newcommand{\disptoday}{\dispkeyval{events-\pgfcalendarshorthand{y}{0}-\pgfcalendarshorthand{m}{0}-\pgfcalendarshorthand{d}{0}}}

\renewcommand{\reminder}[8]
{\pgfcalendar{cal}{#1-#2-#3}{#1-#2-#3}{ %Just to check for correct zero padding
    \addtokey{events-\pgfcalendarshorthand{y}{0}-\pgfcalendarshorthand{m}{0}-\pgfcalendarshorthand{d}{0}}{\dispevent{#1}{#2}{#3}{#4}{#5}{#6}{#7}{#8}}}}

\newcommand{\lesserreminder}[8]
{\pgfcalendar{cal}{#1-#2-#3}{#1-#2-#3}{ %Just to check for correct zero padding
    \addtokey{events-\pgfcalendarshorthand{y}{0}-\pgfcalendarshorthand{m}{0}-\pgfcalendarshorthand{d}{0}}{\displesserevent{#1}{#2}{#3}{#4}{#5}{#6}{#7}{#8}}}}
% reminder
% weekrmdr
\newcommand{\weekrmdr}[8]
{\pgfcalendar{cal}{#1-#2-#3}{#1-#2-#3}{ %Just to check for correct zero padding
    \addtokey{events-\pgfcalendarshorthand{y}{0}-\pgfcalendarshorthand{m}{0}-\pgfcalendarshorthand{d}{0}}{\dispweekreminder{#1}{#2}{#3}{#4}{#5}{#6}{#7}{#8}}}}

\newcommand{\birthday}[4]       %birthday command
{\pgfcalendar{cal}{\weekyear-#1-#2}{\weekyear-#1-#2}{ %Just to check for correct zero padding
    \addtokey{events-\pgfcalendarshorthand{y}{0}-\pgfcalendarshorthand{m}{0}-\pgfcalendarshorthand{d}{0}}{\displesserevent{\weekyear}{#1}{#2}{#3}{#3's birthday \ifthenelse{\equal{#4}{}}{}{%
    \newcount\yearsold
    \yearsold=\weekyear
    \advance\yearsold by -#4
    - \the\yearsold}}
{\ifthenelse{\equal{#4}{}}{}{(#4)}}{}{}}}}

% \renewcommand{\reminder}[8]
% {\addtokey{events-#1-#2-#3}{\dispevent{#1}{#2}{#3}{#4}{#5}{#6}{#7}{#8}}}


\newcounter{loopindex}
\newcounter{calyposition}
\newcounter{weekypos}
\newcounter{weekxpos2}
\newcounter{weekypos2}

\newcount\julianday
\newcount\weekday
\pgfcalendardatetojulian{\theyear-\themonth-\theday}{\julianday}
\pgfcalendarjuliantoweekday{\the\julianday}{\weekday}
\ifnum\weekday>4\relax%
        \advance\weekday by -7\fi
\advance\julianday by -\the\weekday
\pgfcalendarjuliantodate{\the\julianday}{\weekyear}{\weekmonth}{\weekday}
\edef\weekstart{\weekyear-\weekmonth-\weekday}
\edef\weekjulianday{\the\julianday}


\newcount\anothertemp
\pgfcalendardatetojulian{\weekstart+3}{\julianday}
\pgfcalendarjuliantodate{\the\julianday}{\tempyear}{\tempmonth}{\tempday}
\pgfcalendardatetojulian{\tempyear-01-04}{\julianday}
\pgfcalendarjuliantoweekday{\the\julianday}{\anothertemp}
\advance\julianday by -\the\anothertemp
\edef\weekzeroday{\the\julianday}
\julianday=\weekjulianday
\advance\julianday by -\weekzeroday
\divide\julianday by 7
\advance\julianday by 1
\ifthenelse{\the\julianday < 10}{\edef\weeknum{0\the\julianday}}{\edef\weeknum{\the\julianday}}

\newcommand{\ordinaldate}[2]    %Julian date, year
{
  \newcount\julianday
  \pgfcalendardatetojulian{#2-01-01}{\julianday}
  \edef\startofyear{\the\julianday}
  \julianday = #1
  \advance\julianday by -\startofyear
  \advance\julianday by 1
  \edef\firstord{\the\julianday}
  \pgfcalendardatetojulian{#2-12-31}{\julianday}
  \advance\julianday by -#1
  \edef\secondord{\the\julianday}
}

\newlength{\weekxpos}

%%% Local Variables: 
%%% mode: latex
%%% TeX-master: "zeitplanungmaschine.tex"
%%% End: 
