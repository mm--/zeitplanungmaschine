\scriptsize
    \begin{tikzpicture}[xscale=1, yscale=1.97]
      \def\dispweekreminder{}             %Don't display small events
      \def\Tiny{\fontsize{3pt}{3pt}\selectfont}
      \tikzstyle{eventcolor}=[color=black]
      \tikzstyle{eventstyle}=[eventcolor, text width=1cm]
      \let\%=\pgfcalendarshorthand
      \newcount\eventcount
      \newcount\temp
      \renewcommand{\blah}[1]
      {
        \edef\argh{event\pgfcalendarcurrentday-\the\eventcount}
        \advance\eventcount by 1
        \ifthenelse{\equal{\the\eventcount}{1}}
        {  \draw (\&, 5 - \value{calyposition}) ++ (0, -4pt) node (event\pgfcalendarcurrentday-\the\eventcount) [eventstyle] {#1};}
        {  \draw (\argh.south) ++(0, -1pt) node (event\pgfcalendarcurrentday-\the\eventcount) [anchor=north, eventstyle] {#1};}%
      }
      

      \renewcommand{\dispevent}[9]
      {
        \pgfmathsetcount\temp{mod(\pgfcalendarcurrentweekday+1,7)}
        \ifthenelse{\equal{#4}{}}{}
        {
            \blah{\Tiny{#4}}
        }
      }

      \tikzstyle{filling}=[fill=blue!20]

      \draw[help lines] (0,0) grid (7cm, 5cm);

      \setcounter{calyposition}{0}

      \pgfcalendar{cal}{\startdate}{\weekstart+33}
                  {%
                    \leavevmode%
                    \pgfmathparse{mod(\pgfcalendarcurrentweekday+1,7)} % Week begins on Sunday
                    \let\%=\pgfcalendarshorthand
                    \let\&=\pgfmathresult
                    \ifdate{between=\weekstart and \weekstart+6}
                                      {
                                        \tikzstyle{isweek}=[blue]
                                      }{
                                        \tikzstyle{isweek}=[black]
                                        }

                    \ifdate{workday}{\tikzstyle{filling}=[fill=none]}{\tikzstyle{filling}=[fill=gray!20,semitransparent]}
                    \draw (\&, 5 - \value{calyposition}) node [anchor=north west, transform shape, filling, minimum width=1cm, minimum height=1cm] {};
                    \draw (\&, 5 - \value{calyposition}) node (day\pgfcalendarcurrentday) [anchor=north west,isweek] 
                    {\pgfcalendarcurrentday\ifdate{day of month = 1}
                      { \%{m}{.}}{}};
                    %% \draw (\& * 2 + 10, 5 - \value{calyposition}) node [anchor=north] {\offset - \diff - \diffend};

                    \eventcount=0
                    \disptoday

                    \ifdate{between=\weekstart and \weekstart+6}
                           {
                             \draw (\& + 1, 5) node ({\%{w}{.}}) [anchor=south east, filling, fill=none] {\%{w}{.}};
                           }{}

                    %% \ifdate{equals=\weekyear-\weekmonth-\weekday}
                    %%        { 
                    %%        \draw (\&, 5 - \value{calyposition}) node [anchor=north west, green] {\pgfcalendarcurrentday};
                    %%        }{}%

                    \ifdate{Saturday}{ \addtocounter{calyposition}{1}}{}%
                    \ifthenelse{\value{calyposition} > 4}
                                      {
                                        \setcounter{calyposition}{0}
                                      }{}
                  }


                  \pgfcalendar{cal}{\weekyear-\weekmonth-01}{\weekyear-\weekmonth-01}
                              {        
                                \let\%=\pgfcalendarshorthand         
                                \draw node [above of=Wed, node distance=0.3cm] {\%mt \%y0};
                              }

    \end{tikzpicture}

%%% Local Variables: 
%%% mode: latex
%%% TeX-master: "zeitplanungmaschine.tex"
%%% End: 
