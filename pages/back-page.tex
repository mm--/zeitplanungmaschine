\begin{tikzpicture}
   \makeatletter

\newcommand\bigscriptsize{\@setfontsize\
scriptsize{0.75}{0.75}}

% To start on sunday
% Author: Berteun Damman

    % Define our own style
    \tikzstyle{week list sunday}=[
        % Note that we cannot extend from week list,
        % the execute before day scope is cumulative
        execute before day scope={%
               \ifdate{day of month=1}{\ifdate{equals=\pgfcalendarbeginiso}{}{
               % On first of month, except when first date in calendar.
                   \pgfmathsetlength{\pgf@y}{\tikz@lib@cal@month@yshift}%
                   \pgftransformyshift{-\pgf@y}
               }}{}%
        },
        execute at begin day scope={%
            % Because for TikZ Monday is 0 and Sunday is 6,
            % we can't directly use \pgfcalendercurrentweekday,
            % but instead we define \c@pgf@counta (basically) as:
            % (\pgfcalendercurrentweekday + 1) % 7
            \pgfmathsetlength\pgf@x{\tikz@lib@cal@xshift}%
            \ifnum\pgfcalendarcurrentweekday=6
                \c@pgf@counta=0
            \else
                \c@pgf@counta=\pgfcalendarcurrentweekday
                \advance\c@pgf@counta by 1
            \fi
            \pgf@x=\c@pgf@counta\pgf@x
            % Shift to the right position for the day.
            \pgftransformxshift{\pgf@x}
        },
        execute after day scope={
            % Week is done, shift to the next line.
            \ifdate{Saturday}{
                \pgfmathsetlength{\pgf@y}{\tikz@lib@cal@yshift}%
                \pgftransformyshift{-\pgf@y}
            }{}%
        },
        % This should be defined, glancing from the source code.
        tikz@lib@cal@width=7
    ]

    \makeatother
    \def\Tiny{\fontsize{8pt}{8pt}\selectfont}
    \Tiny

\let\%=\pgfcalendarshorthand
 \tikzstyle{year calendar}=[week list sunday, if={(Sunday) [black!70]}, if={(Saturday) [black!70]}, 
 if={(between=\weekstart and \weekstart+6) [blue]}, day xshift=3.2ex,day yshift=1em, month label left vertical, month yshift=1.5em, month xshift=3ex, month text={\ifdate{equals=1-1}{\%m. \%y0}{\%m./\%m0}}];
% \node at (0,0) {hello!};
\matrix [column sep=20pt] at (20pt, -10pt){
 \calendar[dates=\weekyear-01-01 to \weekyear-06-last, year calendar]; & 
 \calendar[dates=\weekyear-07-01 to \weekyear-12-last, year calendar]; \\
};
\end{tikzpicture}

%%% Local Variables: 
%%% mode: latex
%%% TeX-master: "zeitplanungmaschine.tex"
%%% End: 
